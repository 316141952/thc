\documentclass{article} %El formato de documento que utilizo es de artículo.
\usepackage{amsmath} %Este paquete mejora la información estructural y la visualización de documentos que contengan fórmulas matemáticas.
\usepackage{amssymb} %Este paquete provee una colección extendida de símbolos matemáticos.
\usepackage{graphicx} %Este paquete es para poder insertar imágenes
\usepackage[utf8]{inputenc} %Este paquete es para los carácteres especiales en español, como los acentos.
\graphicspath{{/home/Valeria/thc/Clases/Latex/Imagenes}}
\renewcommand*\ttdefault{cmvtt}
\renewcommand*\familydefault{ttdefault}
\usepackage[T1]{fontenc}
\usepackage{hyperref}%para poner links

\title{\Huge Taller de Herramientas Computacionales}
\author{\huge Valeria Ortiz Cervantes}
\date{\LARGE 24 de enero del 2019}

\begin{document}
\maketitle
\begin{center}
	\subsection*{\LARGE Universidad Nacional Autónoma de México.\\Facultad de Ciencias.\\}
	\includegraphics[scale=3]{/1.jpg}
\end{center}
\newpage
\title{\LARGE Tercer Problema de las Tareas 4 y 5.}
\section*{\large Tarea 4}
En esta función lo que hice fue que primero debes meter si lo que quieres convertir son grados Celsius o Farenheit, escrito en una cadena, junto con el número de grados. Ya en el script preguntaba el tipo de grados y después cuántos grados son.\\
La investigación que realicé fue buscar las fórmulas para convertir los grados, esta fue la página de donde lo saqué: 
\url https://www.saberespractico.com/ciencia/como-pasar-de-grados-celsius-a-farenheit-y-viceversa/	
\section*{\large Tarea 5}
Aquí hice dos funciones, una que te daba una lista con los valores de los grados centígrados para convertir, en la entrada tenías que dar el inicio, el final y subintervalos para la lista. Usando eso te regresaba otra lista con los grados convertidos.
\end{document}
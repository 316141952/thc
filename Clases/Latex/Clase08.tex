\documentclass{article}
\usepackage{amsmath}
\usepackage{amssymb}
\usepackage{graphicx}
%\usepackage{enumitem} 
\usepackage[utf8]{inputenc}
\title{\Huge Taller de Herramientas Computacionales}
\author{Valeria Ortiz Cervantes}
\date{16 de enero del 2019}

\begin{document}
	\maketitle
	\newpage
	\title{Clase VIII. Bitácora\\}
	\text{Los temas vistos en clase fueron:}
	\begin{enumerate}
		\item Cosas vistas en Python:
		\begin{enumerate}
			\item float : convertir a flotante.
		    \item int : convertir a entero.
		    \item str : convertir a cadena.
		    \item + : une cadenas.
		    \item ; :finalizador de instrucción.
		    \item type : para saber la clase de un objeto.
		    \item Primeras dos líneas en cada archivo:
		    \begin{verbatim}
		    	%\item $#!/usr/bin/python2.7$
		    	%\item $# -*- coding: utf-8 -*-$
		    \end{verbatim}
		\end{enumerate}
		\item Comandos: 
		\begin{enumerate}
			\item ctrl + z : detener un proceso.
			\item bg : pasa a ejecutar un programa en segundo plano.
			\item $\&$ : ejecuta directamente un programa en segundo plano. 
			\item ctrl + c :termina la ejecución de un programa.
			\item fg : regresar un programa al primer plano.
			\item kill : enviar una señal a un proceso.
			\item kill -9 : para que un proceso se detenga.
			\item top : muestra procesos en tiempo real.
			\item q : salir.
			\item chmod +x : agregar permisos.
			\item find . -name "*abc" : buscar en el directorio actual todos los archivos que tengan en su nombre los carácteres abc.
		\end{enumerate}
		\item Cosas vistas en Latex :
		\begin{enumerate}
			\item Matrices: con begin pones bmatrix y pones los elementos separados por un $\&$. 
			\item Puntos suspensivos :
			\begin{enumerate}
				\item dots : puntos horizontales.
				\item vdots : puntos verticales.
				\item ddots : puntos en diagonal. 
			\end{enumerate} 
			\item Suma (con la letra Sigma mayúscula): sum 
		\end{enumerate}
	\end{enumerate}
\end{document}
\documentclass{article}
\documentclass[letterpaper, 12pt, oneside]{article} %para dar formato al documento
\usepackage{amsmath}
\usepackage{graphicx}
\usepackage{xcolor}
\usepackage{enumitem}
\usepackage[utf8]{inputenc}
\title{\Huge Taller de Herramientas Computacionales}
\author{Valeria Ortiz Cervantes}
\date{8 de Enero del 2019}
\begin{document}
	\maketitle
	\newpage
	\title{Clase II\\Bitácora\\}
	\text{Los temas vistos en la segunda clase fueron:}
	\begin{enumerate}
		\item Github:\\ es un servidor de git, que permite subir, bajar e intercambiar información.
		\item  PATH:\\ Variable de entorno, lugar donde se buscan las variables binarias.
		\item Comandos:
		\begin{enumerate}
			\item top: arroja información sobre la computadora como CPU's y su actividad.
			\item cd
			\begin{enumerate}
				\item  / : directorio raíz 
				\item lib 64 : bibliotecas de 64 bits
				\item lib : bibliotecas
				\item home : usuarios
				\item media : lector USB o CD
				\item mnt : webcams, discos duros externos
				\item . : directorio anterior
				\item .. : directorio anteanterior
				\item /der : aparatos o partes conectadas con el equipo 
			\end{enumerate}
			\item df -lh : qué parte del disco duro se puede utilizar. 
			\item less "nombre" : muestra el contenido de un archivo paginado. 
			\item file "nombre" despliega los archivos de otros archivos o de un directorio. 
			\item Comandos para instalar Git:
			\begin{enumerate}
				\item sudo apt-dnf upgrade
				\item sudo apt-dnf install git
				\item git config --global user.mail "correo electrónico"
				\item git config --global user.name "cuenta"
			\end{enumerate}
			\item git init : convierte una carpeta en un repositorio.
			\item git clone "link del repositorio" : permite añadir un repositorio de Github a la computadora.
			\item git status : muestra los cambios realizados dentro de un repositorio.
			\item git add * : añade todas las acutalizaciones realizadas.
			\item git commit : permite añadir comentarios acerca de la actualizaciones realizadas.
			\item git push : sube las actualizaciones realizadas a la nube.
			\item git pull : baja actualizaciones de la nube.
		\end{enumerate}
	\end{enumerate}
\end{document}
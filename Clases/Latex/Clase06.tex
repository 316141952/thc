%\documentclass{article}
\documentclass[letterpaper, 12pt, oneside]{article} %para dar formato al documento
\usepackage{amsmath}
\usepackage{graphicx}
\usepackage{xcolor}
\usepackage{enumitem}
\usepackage[utf8]{inputenc}
\title{\Huge Taller de Herramientas Computacionales}
\author{Valeria Ortiz Cervantes}
\date{11 de Enero del 2019}
\begin{document}
	\maketitle
	\newpage
	\title{Clase VI. Bitácora\\}
	\text{Los temas vistos en la sexta clase fueron:}
	\begin{enumerate}
		\item Vimos un problema que involucraba el obtener la raíz cuadrada de una variable a partir de dos cuadriláteros usando solo operaciones aritméticas básicas
		\item Definimos una función en Python que nos daba el valor absoluto de un número.
		\item Hicimos un ejercicio nosotros que involucraba una diana infinita en un plano cartesiano y teníamos que definir una función que nos calculara la puntuación según las coordenadas.
		\item Comandos:
		\begin{enumerate}
			\item if : una condición.
			\item else : el caso en que la condición de if no sea verdadera.
			\item and : y
			\item or : o
			\item while : permite ejecutar ciclos o bucles.
		\end{enumerate} 
	\end{enumerate}
\end{document}
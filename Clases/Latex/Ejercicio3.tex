\documentclass{book}
\usepackage[spanish]{babel} 
\usepackage[utf8]{inputenc}
\usepackage{hyperref} %paquete para enlazar cosas y poner links 
\usepackage{biblatex} %paquete para bibliografía

\title{Taller de Herramientas Computacionales}
\author{Valeria Ortíz Cervantes}
\date{17 de enero del 2018}

\begin{document}
\maketitle
%Aquí inicia el Índice
\tableofcontents
\section*{Introducción}Este libro es para fortalecer el conocimiento de la materia Taller de Herramientas Computacionales.
\url{www-google.com}
\hyperref[Google]{www.google.com}
\chapter{Uso básico de Linux.}
\section{Distribuciones de Linux}
\section{Comandos}
\chapter {Introducción a LateX}
\chapter{Introducción a Python}
\begin{verbatim}
	#!/usr/bin/python2.7
	# -*- coding: utf-8 -*-
	'''
	Valeria Ortiz Cervantes, 316141952
	Taller de Herramientas Computacionales
	Este programa fue para aprender cómo ejecutar un programa de python en bash.
	''' 
	x = 10.5; y = 1.0/3; z = 15.3
	#x,y,z = 10.5, 1.0/3, 15.3
	H = '''
	El punto en R3 es :
	(x,y,z) = (%.2f,%g,%G)
	''' % (x,y,z) 
	print H
	
	G = '''
	El punto en R3 es:
	(x,y,z) = ({laX:.2f}, {laY:g}, {laZ:G})
	''' .format(laX = x,laY = y,laZ = z)
	print G
	
	x=input("Cuál es el valor al que le quieres calcular la raíz")
	import math as m
	from math import sqrt
	from math import sqrt as s
	from math import * #no se recomienda
	print 'la raiz cuadrada de %2.f es %f' % (x,m.sqrt(x))
	
	print sqrt (16.5)
	print s (16.5)
\end{verbatim}

\input{/home/thc/thc/Clases/Latex/Prueba.py}\\
\input{Prueba.py}

%Aquí inician los capítulos del libro
\chapter{Introducción a Latex}
\chapter{Introducción a Python}
\section*{Orientación a objetos}

\begin{thebibliography}{9}
	%\bibitem{Computación}
	Autor 
	\textit{algo}
	kwejfn 2019
\end{thebibliography}
\end{document}
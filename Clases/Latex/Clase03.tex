%\documentclass{article}
\documentclass[letterpaper, 12pt, oneside]{article} %para dar formato al documento
\usepackage{amsmath}
\usepackage{graphicx}
\usepackage{xcolor}
\usepackage{enumitem}
\usepackage[utf8]{inputenc}
\title{\Huge Taller de Herramientas Computacionales}
\author{Valeria Ortiz Cervantes}
\date{9 de Enero del 2019}
\begin{document}
	\maketitle
	\newpage
	\title{Clase III\\Bitácora\\}
	\text{Los temas vistos en la tercera clase fueron:}
	\begin{enumerate}
		\item Comandos: 
		\begin{enumerate}
			\item file "nombre" : indica el tipo de archivo.
			\item ls -la : muestra información adicional sobre los directorios .
			\item cat "nombre" : muesra el contenido de un archivo.
			\item esc + shift + : 
			\begin{enumerate}
				\item wq : guardar y salir.
				\item q! : salir sin guardar.
			\end{enumerate}
			\item mkdir -p : para crear directorios alienados, uno dentro de otro.
			\item git commit -m "comentario" : para saltarnos el ir a vi para escribir el comentario.
			\item touch README.md : crea un documento readme.
			\item history : muestra la historia de comandos utilizados.
		\end{enumerate}
		\item La resolución de un problema:
		\begin{enumerate}
			\item Definir y entender el problema.
			\item Analizar y delimitar.
			\item Soluciones posibles.
			\item Describir soluciónco detalle.
			\item Solución general.
		\end{enumerate}
	\end{enumerate}
\end{document}
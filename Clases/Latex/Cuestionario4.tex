\documentclass{article} %El formato de documento que utilizo es de artículo.
\usepackage{amsmath} %Este paquete mejora la información estructural y la visualización de documentos que contengan fórmulas matemáticas.
\usepackage{amssymb} %Este paquete provee una colección extendida de símbolos matemáticos.
\usepackage{graphicx} %Este paquete es para poder insertar imágenes
\usepackage[utf8]{inputenc} %Este paquete es para los carácteres especiales en español, como los acentos.
\graphicspath{{/home/Valeria/thc/Clases/Latex/Imagenes}}


\title{\Huge Taller de Herramientas Computacionales}
\author{\huge Valeria Ortiz Cervantes}
\date{\LARGE 23 de enero del 2019}


\begin{document}
\maketitle
\begin{center}
	\subsection*{\LARGE Universidad Nacional Autónoma de México.\\Facultad de Ciencias.\\}
	\includegraphics[scale=3]{/1.jpg}
\end{center}
\newpage
\title{\LARGE Cuestionario de la bitácora XIII.}
\begin{enumerate}
	\item ¿De qué manera puedes saber si una lista es vacía o no para usarla en una condicional?\\Usando if bool("nombre"): o if "nombre":
	\item ¿Cómo puedes omitir el salto de línea después de un print?\\Poniendo una coma al final.
	\item ¿Qué es una función recursiva?\\Son funciones que se vuelvena  llamar a si mismas; se dividen en dos partes, una que es el caso base y otra que es la parte recursiva que llama de nuevo a la función.
	\item ¿Cuántos tipos de variables hay?\\Variables Locales y variables Globales.
	\item ¿Qué es una variable Local?\\ Una variable que se encuentra dentro de una función, al estar dentro de una función sólo es válida dentro de la misma.
	\item ¿Qué es una variable Global?\\Una variable que se encuentra fuera de una función, esta es válida tanto dentro como fuera de funciones.
\end{enumerate}
\end{document}
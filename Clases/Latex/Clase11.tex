\documentclass{article} %El formato de documento que utilizo es de artículo.
\usepackage{amsmath} %Este paquete mejora la información estructural y la visualización de documentos que contengan fórmulas matemáticas.
\usepackage{amssymb} %Este paquete provee una colección extendida de símbolos matemáticos.
\usepackage{graphicx} %Este paquete es para poder insertar imágenes
\usepackage[utf8]{inputenc} %Este paquete es para los carácteres especiales en español, como los acentos.
\graphicspath{{/home/Valeria/thc/Clases/Latex/Imagenes}}

\title{\Huge Taller de Herramientas Computacionales}
\author{Valeria Ortiz Cervantes}
\date{21 de enero del 2019}

\begin{document}
	\maketitle
	\begin{center}
		\subsection*{Universidad Nacional Autónoma de México.\\Facultad de Ciencias.\\}
		%\includegraphics[scale=3]{/1.jpg}
	\end{center}
	\newpage
	\subsection*{Clase XI. Bitácora\\}
	\text{En la clase de hoy solo vimos cosas de Python, a continunación pondré los temas vistos y las funciones nuevas que aprendimos:}
	\begin{enumerate}
		\item Vimos un ejemplo de cómo sacar el promedio de los elementos de una lista, era usando el comando for y con la función len, pero convirtiendo uno de los valores a flotante. 
		\item Vimos cómo incrementar todos los elementos de una lista en un valor determinado: utilizando el comando for y las funciones range y len, una dentro de otra de la siguiente forma range(len("nombre")).
		\item Hicimos un programa en el que definimos varias funciones para resolver el problema 2 de la tarea 5,la conversión de grados Celsius a Farenheit, usando listas.
		\item Vimos una nueva función, zip, que es para hacer duplas entre listas que contengan la misma cantidad de elementos.
		\item Al final dejó un ejercicio de hacer una lista de listas, con la función zip, si me salió pero como me dormí en una explicación lo hice muy largo.
	\end{enumerate}
\end{document}
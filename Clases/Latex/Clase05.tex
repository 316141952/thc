%\documentclass{article}
\documentclass[letterpaper, 12pt, oneside]{article} %para dar formato al documento
\usepackage{amsmath}
\usepackage{graphicx}
\usepackage{xcolor}
\usepackage{enumitem}
\usepackage[utf8]{inputenc}
\title{\Huge Taller de Herramientas Computacionales}
\author{Valeria Ortiz Cervantes}
\date{11 de Enero del 2019}
\begin{document}
	\maketitle
	\newpage
	\title{Clase V\\Bitácora\\}
	\text{Los temas vistos en la quinta clase fueron:}
	\begin{enumerate}
		\item Cosas vistas en Python:
		\begin{enumerate}
			\item Cadenas: son una serie de carácteres delimitados por comillas.
			\begin{enumerate}
				\item Cadena de una sola línea: con una comilla o dos (', ").
				\item Cadena multilínea: con tres comillas (''').
			\end{enumerate}
			\item Valor de una variable: con el signo de porcentaje ($\%$) y una letra despliegas el valor de una variable. 
			\begin{enumerate}
				\item $\%$g : mostrar la variable en el formato numérico más corto posible.
				\item $\%$f : muestra el valor numérico con punto flotante, con 4 decimales que es lo estándar.\\Ejemplo: $\%$10.2f; esto quiere decir que la variable con dos decimales y con espacios para 10 cifras.
				\item $\%$e : mostrar la variable con notación científica. 
				\item $\%$E : mostrar la variable con notación científica pero con e en el exponente.
				\item $\%$d : se muestra solo el número entero.
				\item $\%$s : el valor de la variable es una cadena, no un valor numérico.
			\end{enumerate}
			\item Palabras reservadas: son aquellas que tienen un significado exclusivo dentro del lenguaje de programación.\\Palabras reservadas en Python:
			\begin{enumerate}
				\item print
				\item import
				\item def
				\item return
			\end{enumerate}
			\item Módulo math: un módulo que tiene como funciones operaciones algebraicas y trigonométricas, así como el valor de un par de constantes. 
			\begin{enumerate}
				\item sqrt : raíz cuadrada 
				\item pow : potencia 
				\item e : constante e
				\item pi : constante pi
				\item tau : constante tau
				\item ceil : función techo
				\item floor : función piso
				\item etc.
			\end{enumerate}
		\end{enumerate}
	    \item Cosas vistas en Latex, vimos cómo:
	    \begin{enumerate}
	    	\item Crear un documento
	    	\item Agregar paquetes.
	    	\item Poner título, nombre y fecha. 
	    	\item Poner un texto.
	    	\item Hacer un listado, incluso un listado dentro de otro.
	    	\item Colocar una imagen (png, jpg).
	    \end{enumerate}
	\end{enumerate}
\end{document}
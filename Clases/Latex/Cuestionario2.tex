\documentclass{article} %El formato de documento que utilizo es de artículo.
\usepackage{amsmath} %Este paquete mejora la información estructural y la visualización de documentos que contengan fórmulas matemáticas.
\usepackage{amssymb} %Este paquete provee una colección extendida de símbolos matemáticos.
\usepackage{graphicx} %Este paquete es para poder insertar imágenes
\usepackage[utf8]{inputenc} %Este paquete es para los carácteres especiales en español, como los acentos.
\graphicspath{{/home/Valeria/thc/Clases/Latex/Imagenes}}
\usepackage{courier}
\renewcommand*\familydefault{\ttdefault}

\title{\Huge Taller de Herramientas Computacionales}
\author{\huge Valeria Ortiz Cervantes}
\date{\LARGE 21 de enero del 2019}


\begin{document}
\maketitle
\begin{center}
	\subsection*{\LARGE Universidad Nacional Autónoma de México.\\Facultad de Ciencias.\\}
	\includegraphics[scale=3]{/1.jpg}
\end{center}
\newpage
\title{\LARGE Cuestionario de la bitácora XI.}
\begin{enumerate}
	\item ¿Cómo puedes recorrer cada elemento en una lista?\\Con: for i in "nombre de la lista".
	\item ¿Cómo puedes recorrer los elementos de la lista yendo por los índices?\\Con: for i in range (len ("nombre de la lista") ).
	\item ¿Cómo puedes sumarle un número a todos los elementos de la lista a la vez?
	\begin{enumerate}
		\item Con: for x in range (len ("nombre de la lista" ):\\"nombre de la lista"[x] += c
		\item Con: for x,y in enumerate ("nombre de la lista") :\\"nombre de la lista"[x] = y + c
	\end{enumerate}
	\item ¿Qué hace la función zip?\\Crea duplas con elementos de listas, siempre y cuando las listas tengan la misma longitud.
\end{enumerate}
\end{document}
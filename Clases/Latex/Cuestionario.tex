%\documentclass{article}
\documentclass[letterpaper, 12pt, oneside]{article} %para dar formato al documento
\usepackage{amsmath}
\usepackage{graphicx}
\usepackage{xcolor}
\usepackage{enumitem}
\usepackage[utf8]{inputenc}
\title{\Huge Taller de Herramientas Computacionales}
\author{Valeria Ortiz Cervantes}
\date{14 de Enero del 2019}
\begin{document}
	\maketitle
	\newpage
	\title Cuestionario
	\begin{enumerate}
		\item ¿Cuales son los tipos de sistemas operativos mencionados en clase?\\Windows, Linux, IOS.
		\item ¿Cuáles son los lenguajes de programación mencionados en clase?\\Python, Java, C, C+, C++.
		\item ¿Cuál es la definición de bit?\\Una manera de representar la información, puede ser un uno o un cero. 
		\item ¿Cuál es la definición de shell?\\Un shell es un intérprete de comandos especializado, un programa que está a la espera de una instrucción para realizarla.
		\item ¿Con qué comando puedes ver el contenido de un directorio?\\ls 
		\item ¿Cómo puedes ver las variables de entorno?\\Con el comando shell.
		\item ¿Qué hace el comando touch /tmp/"nombre"?\\Crea un archivo vacío.
		\item ¿Cómo puedes acceder a la información sobre los permisos de un archivo?\\Con el comando ls -l /tmp/"nombre".
		\item ¿Cómo puedes saber en qué directorio estás actualmente?\\Con el comando pwd.
		\item ¿Qué es Github?\\Es un servidor de git, que permite subir, bajar e intercambiar información.
		\item ¿Qué hace el comando top?\\arroja información sobre la computadora como CPU's y su actividad.
		\item ¿Cómo se puede acceder a las bibliotecas?\\Con el comando cd lib.
		\item ¿Cómo regresas al directorio anterior?\\Con el comando cd.
		\item ¿Cómo puedes ver qué parte del disco duro se puede utilizar?\\Con el comando df -lh.
		\item ¿Cómo se crea un repositorio?
		\begin{enumerate}
			\item Creando una carpeta en la terminal y después poniendo el comando git init.
			\item Creando el repositorio desde github y después lo subes a la computadora con el comando git clone y el link del repositorio. 
		\end{enumerate}
		\item ¿Cómo puedes ver los cambios realizados en un repositorio?\\Con el comando git status.
		\item ¿Con qué comando agregas todas las actualizaciones al repositorio?\\git add *
		\item ¿Cómo puedes subir las actualizaciones hechas desde la computadora a la nube?\\Con el comando git push, después metes tu ususario y tu contraseña.
		\item ¿Cómo puedes descargar las actualizaciones realizadas por otro equipo?\\Con el comando git pull.
		\item ¿Qué hace el comando ls -la?\\Muestra información adicional del directorio en el que te encuentres.
		\item ¿Con qué comando se crean directorios alienados?\\Con mkdir -p
		\item ¿Cómo puedes comentar las actualizaciones de manera más directa?\\Con el comando git commit -m y el comentario entre comillas.
		\item ¿Con qué comando se muestran los comandos utilizados en la sesión?\\history
		\item ¿Cómo se representan los números reales en el shell de Python?\\Con el sistema de punto flotante.
		\item ¿Qué es una cadena?\\Una serie de carácteres delimitados por comillas
		\item ¿Cuántas comillas se usan para una cadena de una sola línea?\\Una o dos (',").
		\item ¿Cuántas comillas se usan para una cadena multilínea?\\Tres (''').
		\item ¿Cómo despliegas el valor de una variable en python?\\Con el signo de porcentaje acompañado de una letra.
		\item ¿Cómo muestras sólo la parte entera de una variable?\\Con $\%$d
		\item ¿Cómo expresas una variable en notación científica?\\Con $\%$e o $\%$E
		\item ¿Cómo expresas una variable con 3 decimales?\\Con $\%$.3f
		\item ¿Cómo expresas una variable de la forma más corta posible?\\Con $\%$g
		\item ¿Cómo expresas el valor de una variable en una cadena?\\Con $\%$s
		\item ¿Qué son las palabras reservadas?\\Son aquellas que tienen un significado exclusivo dentro del lenguaje de programación.
		\item ¿Cuáles son las palabras reservadas de Python vistas en clase?
		\begin{enumerate}
			\item print
			\item import
			\item def
			\item return
		\end{enumerate}
		\item ¿Cómo sacas una raíz cuadrada en Python?\\Importas en módulo math, después con math.sqrt y pones el número entre paréntesis.
	\end{enumerate}
\end{document}
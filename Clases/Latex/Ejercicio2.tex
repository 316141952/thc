\documentclass{article}
\usepackage{amsmath}
\usepackage{amssymb}
\usepackage{graphicx}
\usepackage{enumitem} %poder ocupar los paquetes para enlistar
\usepackage[utf8]{inputenc}%carácteres especiales del español (acentos)
\graphicspath{{/home/thc/Imagenes}}

\title{\Huge Taller de Herramientas Computacionales}
\author{Valeria Ortiz Cervantes}
\date{15 de enero del 2019}

\begin{document}
	\maketitle
\begin{center}
	\includegraphics[scale=0.50]{/home/thc/Imagenes/1.png}
\end{center}
\newpage
\section*{Expresiones Matemáticas}
$\alpha +  \beta$\\ %los signos de pesitos son para escribir expresiones matemáticas
\[\alpha + \beta\] %los corchetes centran la expresion y en una sola línea
\section*{Índices y subíndices}
$x_{2}$ % x subíndice 2
$x^{2}$ \\ % superíndice, exponentes

$\frac{2}{7}$ %escribir fracciones
$\frac{\frac{3}{4}}{\frac{2}{5}}$

$\sqrt{2} + \sqrt{3^2}^2$ \\
\%   $\%$

$\int_{a}^{b} x^2 dx$ \\ 
$\int_{a}^{b} x^2 \partial$
\end{document}
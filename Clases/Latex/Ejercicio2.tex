\documentclass{article}
\usepackage{amsmath}
\usepackage{amssymb}
\usepackage{graphicx}
%\usepackage{enumitem} poder ocupar los paquetes para enlistar
\usepackage[utf8]{inputenc}%carácteres especiales del español (acentos)
\graphicspath{{/home/thc/Imagenes}}

\title{\Huge Taller de Herramientas Computacionales}
\author{Valeria Ortiz Cervantes}
\date{15 de enero del 2019}

\begin{document}
	\maketitle
%\begin{center}
%	\includegraphics[scale=0.50]{/home/thc/Imagenes/1.png}
%\end{center}
\newpage
\section*{Expresiones Matemáticas}
$\alpha +  \beta$\\ %los signos de pesitos son para escribir expresiones matemáticas
\[\alpha + \beta\] %los corchetes centran la expresion y en una sola línea
\section*{Índices y subíndices}
$x_{2}$ % x subíndice 2
$x^{2}$ \\ % superíndice, exponentes

$\frac{2}{7}$ %escribir fracciones
$\frac{\frac{3}{4}}{\frac{2}{5}}$

$\sqrt{2} + \sqrt{3^2}^2$ \\
\%   $\%$

$\int_{a}^{b} x^2 dx$ \\ 
$\int_{a}^{b} x^2 \partial$
\section*{Matrices}
%dots puntos suspensivos
%vdots puntos suspensivos verticales 
\[
\begin{bmatrix}
	x_{2} & x_{3}\\
	x_{4} & x_{7}
\end{bmatrix}
\]
\[ %en una sola línea
\begin{bmatrix}
x_{2} & x_{5} & \dots\\
x_{5} & x_{20} & \ddots\\
\vdots & \vdots & \vdots
\end{bmatrix}
\]
$\sum$

\section*{Tablas}
\[
\begin{array}{|c|c|c|}
\hline
f(t) & F(S) & \mbox{Remark}\\
\hline \hline
\delta(t) & 1 & \mbox{impulse function}\\
u(t) & \frac{1}{s} & \mbox{unit step function}\\
e^{at}u(t) & \frac{1}{s-a} & \mbox{one-side exponential}\\
\hline
\end{array}
\]

\section*{Alineamiento}
\begin{align*}
	2x - 5y &= 8 \\
	2 - 9y &= -12
\end{align*}
\end{document}
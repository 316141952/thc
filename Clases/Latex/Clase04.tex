%\documentclass{article}
\documentclass[letterpaper, 12pt, oneside]{article} %para dar formato al documento
\usepackage{amsmath}
\usepackage{graphicx}
\usepackage{xcolor}
\usepackage{enumitem}
\usepackage[utf8]{inputenc}
\title{\Huge Taller de Herramientas Computacionales}
\author{Valeria Ortiz Cervantes}
\date{10 de Enero del 2019}
\begin{document}
	\maketitle
	\newpage
	\title{Clase IV\\Bitácora\\}
	\text{Los temas vistos en la cuarta clase fueron:}
	\begin{enumerate}
		\item El problema de la pelota:\\Realizamos los pasos acerca de la resolución de un problema en el problema de una pelota y después pasamos a programar algo acerca de ello en el shell de python.
		\item Sistema de punto flotante:\\La representación de los números reales.
		\item idle\\ Es un IDE, entorno de programación integrado. Es un shell de Python.
		\item Comandos:
		\begin{enumerate}
			\item python : al darle dos veces en el tabulador te muestra las versiones disponibles de python.
			\item python --version : muestra la versión de python que se utiliza.
			\item idle : 
			\begin{enumerate}
				\item Con doble click al tabulador muestra las versiones disponibles de idle.
				\item Solito y dándole enter abre el shell de python.
				\item print : mostrar (comando en python).
			\end{enumerate}
		\end{enumerate}
			\item Operaciones aritméticas en Python:
			\begin{enumerate}
				\item + : suma
				\item - : resta 
				\item / : división
				\item * : multiplicación
				\item ** : potencia
			\end{enumerate}
	\end{enumerate}
\end{document}
%\documentclass{article}
\documentclass[letterpaper, 12pt, oneside]{article} %para dar formato al documento
\usepackage{amsmath}
\usepackage{graphicx}
\usepackage{xcolor}
\usepackage{enumitem}
\usepackage[utf8]{inputenc}
\title{\Huge Taller de Herramientas Computacionales}
\author{Valeria Ortiz Cervantes}
\date{7 de Enero del 2019}
\begin{document}
	\maketitle
	\newpage
	\title{Clase I\\Bitácora\\}
	\text{Los temas vistos en la primera clase fueron:}
	\begin{enumerate}
		\item Sistemas operativos:\\Windows, Linux, IOS, etc.
		\item Lenguajes de programación:
		\begin{enumerate}
			\item Python
			\item Java
			\item C
			\item C+
			\item C++
		\end{enumerate}
	\item Definición de bit:\\Una manera de representar la información, puede ser un uno o un cero. 
	\item Definición de shell:\\ un shell es un intérprete de comandos especializado, un programa que está a la espera de una instrucción para realizarla. 
		\item Comandos:
	\begin{enumerate}
		\item ls : muestra el contenido de un directorio.
		\item set : te permite ver las variables de entorno.
		\item pwd : muestra en qué direcotorio estoy actualmente.
		\item cd : \begin{enumerate}
			\item solito te regresa a home. 
			\item agregando el nombre de algun directorio o carpeta, te manda a él.                 
		\end{enumerate}  
		\item touch /tmp/"nombre" : crea un archivo vacío.
		\item ls -l /tmp/"nombre" : arroja información sobre permisos del archivo /tmp/"nombre"
	\end{enumerate}                         
	\end{enumerate}
\end{document}

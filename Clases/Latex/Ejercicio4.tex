\documentclass{beamer} %hacer presentaciones
\usepackage{graphicx}
\usepackage[utf8]{inputenc}
\usepackage[spanish]{babel}
\graphicspath{{/home/thc/Imagenes}}
%\usetheme{CambridgeUS}
%\usetheme{Bergen}
%\usetheme{Hannover} este es como goettingen pero con el cosito lila a la izquierda
%\usetheme{Goettingen}

%Esto únicamente se utiliza para el tema Bergen
%\def\insertauthorindicator{¿Quién?}
%\def\inserdateindicator {Fecha}

\title{Taller de Herramientas Computacionales}
\author{Valeria Ortiz Cervantes}
\date{22 de enero del 2019}

\begin{document}
\maketitle
\begin{frame}
	\transdissolve
	\frametitle{Mi primera presentación en LateX}
		\begin{center}
			\includegraphics[scale=0.50]{/home/thc/Imagenes/1.png}
		\end{center}
\end{frame}

\begin{frame}
\transblindsvertical
	\frametitle{Segunda diapositiva}
	Esta es mi segunda diaposiiva.
\end{frame}

\begin{frame}[fragile] %para meter texto con caracteres especiales
\transdissolve 
	\begin{verbatim}
		#!/usr/bin/python2.7
		# -*- coding: utf-8 -*-
		'''
		Valeria Ortiz Cervantes, 316141952
		Taller de Herramientas Computacionales
		Aquí vimos distintas fomas de ponerle valor a una variable.
		Vimos cómo importar algo y renombrarlo.
		Y vimos la función input.
		''' 
		x = 10.5; y = 1.0/3; z = 15.3
		#x,y,z = 10.5, 1.0/3, 15.3
		H = '''
		El punto en R3 es :
		(x,y,z) = (%.2f,%g,%G)
		''' % (x,y,z) 
		print H
		
		G = '''
		El punto en R3 es:
		(x,y,z) = ({laX:.2f}, {laY:g}, {laZ:G})
		''' .format(laX = x,laY = y,laZ = z)
		print G
		
		x=input("Cuál es el valor al que le quieres calcular la raíz")
		import math as m
		from math import sqrt
		from math import sqrt as s
		from math import * #no se recomienda
		print 'la raiz cuadrada de %2.f es %f' % (x,m.sqrt(x))
		
		print sqrt (16.5)
		print s (16.5)
	\end{verbatim}
\end{frame}
\end{document}

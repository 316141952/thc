\documentclass{article}
\usepackage{amsmath}
\usepackage{amssymb}
\usepackage{graphicx}
%\usepackage{enumitem} 
\usepackage[utf8]{inputenc}
\title{\Huge Taller de Herramientas Computacionales}
\author{Valeria Ortiz Cervantes}
\date{18 de enero del 2019}

\begin{document}
	\maketitle
	\newpage
	\title{Clase X. Bitácora\\}
	\text{Los temas vistos en clase fueron:}
	\begin{enumerate}
		\item Comandos :
		\begin{enumerate}
			\item ; : separa comandos.
			\item "comando" $\&$ "comando" : ejecuta comandos simultáneamente y en segundo plano.
			\item not : no
		\end{enumerate}
		\item Cosas vistas en Python :
		\begin{enumerate}
			\item bool : te dice si una cadena es verdadera (si tiene carácteres) o falsa (si está vacía); en valores numéricos es falso solo si el valor es cero.
			\item $[$ $]$ : para delimitar una lista.
			\item "nombre".append(x) : para agregar un elemento al final de una lista.
			\item "nombre"$[$x$]$ : para mostrar un elemento, con el nombre de la lista e índice del elemento que quieres ver.
			\item len("nombre") : para ver el número de elementos de una lista.
			\item "nombre".insert(índice,objeto) : agregar un elemento en un índice específico. 
			\item "nombre".pop(índice) : sacar elementos y mostrarlos, le pones el índice del elemento que quieres que saque; si no pones el número saca el último elemento, y se puede almacenar el valor en una variable para ser utilizado después.
			\item "nombre".extend([lista]) : para agregar elementos de una lista a otra.
			\item range[núm.inicial,límite,incremento] : para crear una lista a partir de un cierto número, con un límite y un cierto incremento.
			\end{enumerate}
	\end{enumerate}
\end{document}
\documentclass{article} %El formato de documento que utilizo es de artículo.
\usepackage{amsmath} %Este paquete mejora la información estructural y la visualización de documentos que contengan fórmulas matemáticas.
\usepackage{amssymb} %Este paquete provee una colección extendida de símbolos matemáticos.
\usepackage{graphicx} %Este paquete es para poder insertar imágenes
\usepackage[utf8]{inputenc} %Este paquete es para los carácteres especiales en español, como los acentos.
\graphicspath{{/home/Valeria/thc/Clases/Latex/Imagenes}}
\renewcommand*\ttdefault{cmvtt}
\renewcommand*\familydefault{ttdefault}
\usepackage[T1]{fontenc}
\usepackage{hyperref}%para poner links

\title{\Huge Taller de Herramientas Computacionales}
\author{\huge Valeria Ortiz Cervantes}
\date{\LARGE 24 de enero del 2019}

\begin{document}
\maketitle
\begin{center}
	\subsection*{\LARGE Universidad Nacional Autónoma de México.\\Facultad de Ciencias.\\}
	\includegraphics[scale=3]{/1.jpg}
\end{center}
\newpage
\title{\LARGE Primer Problema de Python fácil.\\}
Para este problema hice lo que nos enseñó el profesor para resolver para resolver un problema, lo definí y después usé ejemplos pequeños para ir creando la función entorno a ellos, resultó muy bien, primero lo escribí en papel, después lo pasé a python y me salió bien a la primera, creo que lo único que me falló fue que no había puesto lo de range(len()).
\end{document}
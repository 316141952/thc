\documentclass{article} %El formato de documento que utilizo es de artículo.
\usepackage{amsmath} %Este paquete mejora la información estructural y la visualización de documentos que contengan fórmulas matemáticas.
\usepackage{amssymb} %Este paquete provee una colección extendida de símbolos matemáticos.
\usepackage{graphicx} %Este paquete es para poder insertar imágenes
\usepackage[utf8]{inputenc} %Este paquete es para los carácteres especiales en español, como los acentos.
\graphicspath{{/home/Valeria/thc/Clases/Latex/Imagenes}}
\renewcommand*\ttdefault{cmvtt}
\renewcommand*\familydefault{ttdefault}
\usepackage[T1]{fontenc}

\title{\Huge Taller de Herramientas Computacionales}
\author{\huge Valeria Ortiz Cervantes}
\date{\LARGE 24 de enero del 2019}

\begin{document}
\maketitle
\begin{center}
	\subsection*{\LARGE Universidad Nacional Autónoma de México.\\Facultad de Ciencias.\\}
	\includegraphics[scale=3]{/1.jpg}
\end{center}
\newpage
\title{\LARGE Segundo Problema de las Tareas 4 y 5.}
\section*{\large Tarea 4}
	Lo primero que hice fue despejar el tiempo de la función que teníamos del movimiento de la pelota, con el ejemplo que habíamos visto anteriormente, usé la fórmula general para funciones cuadráticas y de ahí me salió la función para obtener ambos tiempos.\\	
	Después definí dos funciones para cada uno de los tiempos, para que funcionara bien con el script que quería hacer.
\section*{\large Tarea 5}
	En esta parte, en lo que está comentado fue una función que traté de hacer que tenía la velocidad, la altura y una tupla con ambos tiempos, pero no me salió, así que mejor lo cambié e hice una función que me daba una lista con ambos tiempos y después otra función que me mostraba esos tiempos como una tabla.
\end{document}
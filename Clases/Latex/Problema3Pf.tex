\documentclass{article} %El formato de documento que utilizo es de artículo.
\usepackage{graphicx} %Este paquete es para poder insertar imágenes
\usepackage[utf8]{inputenc} %Este paquete es para los carácteres especiales en español, como los acentos.
\graphicspath{{/home/Valeria/thc/Clases/Latex/Imagenes}}
\renewcommand*\ttdefault{cmvtt}
\renewcommand*\familydefault{ttdefault}
\usepackage[T1]{fontenc}
\usepackage{hyperref}%para poner links

\title{\Huge Taller de Herramientas Computacionales}
\author{\huge Valeria Ortiz Cervantes}
\date{\LARGE 24 de enero del 2019}

\begin{document}
\maketitle
\begin{center}
	\subsection*{\LARGE Universidad Nacional Autónoma de México.\\Facultad de Ciencias.\\}
	\includegraphics[scale=3]{/1.jpg}
\end{center}
\newpage
\title{\LARGE Tercer Problema de Python fácil.\\}
Bueno, resulta que este problema ya lo había resuelto para la Tarea 4, puse que fueran divisibles entre los números del 2 al 10, no puse el 1 pues todos los números son divisibles entre 1, creo que el único problema que tuve fue que al meter los valores de 3,5,7 y 9 me salía que era compuesto cuando no lo son, así que solo agregué otro caso para definir que no lo eran.
\end{document}
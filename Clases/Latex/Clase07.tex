\documentclass{article}
\usepackage{amsmath}
\usepackage{amssymb}
\usepackage{graphicx}
\usepackage{enumitem} %poder ocupar los paquetes para enlistar
\usepackage[utf8]{inputenc}%carácteres especiales del español (acentos)
%\graphicspath{{/home/thc/Imágenes}}

\title{\Huge Taller de Herramientas Computacionales}
\author{Valeria Ortiz Cervantes}
\date{15 de enero del 2019}

\begin{document}
	\maketitle
	\newpage
	\title{Clase VII. Bitácora\\}
	\text{Lo que vimos en la séptima clase fue:}
	\begin{enumerate}
		\item Definimos una función que nos daba una raíz cuadrada con ayuda del problema de la clase anterior del cuadrado.
		\item Vimos la sucesión  de Ulam y realizamos un ejercicio en equipos que consistía en hacer una función que debía darnos el número de operaciones que se realizaban a partir de un cierto número. \\ Después el profesor realizó la función y resultó que involucraba crear dos funciones anteriores a esa: una que nos diera el sucesor del número en la sucesión y otra que nos diera los términos de la sucesión a partir de un número. Para al final solo modificar la segunda y que nos diera el número de términos en la sucesión a partir de un número.
		\item Cosas nuevas de Python:
		\begin{enumerate}
			\item \% : muestra el residuo de una división
			\item ! : no 
			\item != : diferente 
			\item = : asignación
			\item == : comparación 
		\end{enumerate}
		\item Cosas vistas en Latex :
		\begin{enumerate}
			\item Cómo poner una imagen.
			\item Cómo escribir expresiones matemáticas:\\(Los signos de pesitos se usan para indicar que es una expresión matemática.)
			\begin{enumerate}
				\item Letras griegas.\\Para poner una letra se escribe su nombre en inglés.
				\item Subíndices y superíndices(exponentes).\\Para el subíndice se usa un guión bajo y para el super índice el gorrito; y en ambos casos el número va entre llaves.
				\item Fracciones.\\Pones frac y ambos números entre llaves diferentes.
				\item Raíz cuadrada.\\Pones sqrt, como en Python, y el número entre llaves.
				\item Integrales definidas.\\Pones int seguido de un guion bajo y pones entre llaves los numeritos separados por un gorrito, al lado la función y el dx
			\end{enumerate}
			\item Los corchetes se usan para centrar algo en una sola línea.
		\end{enumerate}
	\end{enumerate}
\end{document}
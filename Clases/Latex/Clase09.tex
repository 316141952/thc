\documentclass{article}
\usepackage{amsmath}
\usepackage{amssymb}
\usepackage{graphicx}
%\usepackage{enumitem} 
\usepackage[utf8]{inputenc}
\title{\Huge Taller de Herramientas Computacionales}
\author{Valeria Ortiz Cervantes}
\date{17 de enero del 2019}

\begin{document}
	\maketitle
	\newpage
	\title{Clase IX. Bitácora\\}
	\text{Los temas vistos en clase fueron:}
	\begin{enumerate}
		\item Cosas vistas en Latex:
		\begin{enumerate}
			\item Tablas:
			\begin{enumerate}
				\item Se empieza poniendo begin$\{$array$\}$ y end$\{$array$\}$
				\item Pones una variable cualquiera entre llaves y $\|$ para especificar el número de columnas.
				\item Utilizas hline para las líneas horizontales y $\&$ para las verticales.
			\end{enumerate}
			\item Alineamiento: pones begin$\{$align*$\}$ y end$\{$align*$\}$ y escribes lo que quieras poner alineado. 
			\item Cómo hacer un libro:
			\begin{enumerate}
				\item Índice : pones tableofcontents para el índice, el título de los capítulos va con chapter y los subtítulos con section.
				\item Insertar un link: 
				\begin{enumerate}
					\item Con url y el link entre llaves.
					\item Con hyperref, la plataforma del link entre corchetes y el link entre llaves.
				\end{enumerate}
				\item Hacer los capítulos con subtítulos: los pones igual que en el índice pero más abajo.
				\item Insertar un código de python sin modificarlo: 
				\begin{enumerate}
					\item Con begin $\{$verbatim$\}$ y end$\{$verbatim$\}$.
					\item Con input y la dirección del archivo, si están en la misma carpeta solo con el nombre del archivo.
				\end{enumerate}
				\item Bibliografía: con el paquete biblatex y begin$\{$thebibliography$\}$, end$\{$thebibliography$\}$; al lado del begin va un número entre corchetes indicando el número de referencias.
			\end{enumerate}
		\end{enumerate}
		\item Cosas vistas en Python:
		\begin{enumerate}
			\item Resolvimos el primer problema entre todos con una sucesión recursiva.
			\item getwd : tiene la misma función que el comando pwd en la terminal.
			\item listir :tiene la misma función que el comando ls en la terminal.
			\item chdir : tiene la misma función que el comando cd en la terminal.
		\end{enumerate}
	\end{enumerate}
\end{document}
	\end{enumerate}
\end{document}
\documentclass{article}
\usepackage{amsmath}
\usepackage{amssymb}
\usepackage{graphicx}
%\usepackage{enumitem} 
\usepackage[utf8]{inputenc}
\title{\Huge Taller de Herramientas Computacionales}
\author{Valeria Ortiz Cervantes}
\date{17 de enero del 2019}

\begin{document}
	\maketitle
	\newpage
	\title{Clase IX. Bitácora\\}
	\text{Los temas vistos en clase fueron:}
	\begin{enumerate}
		\item Cosas vistas en Latex:
		\begin{enumerate}
			\item Tablas:
			\begin{enumerate}
				\item Se empieza poniendo begin$\{$array$\}$, end$\{$array$\}$.
				\item Seguido de otras llaves con una variable cualquiera separada por el caracter $\|$, que es lo que define el número de columnas.
				\item Se utiliza el caracter $\&$ para escribir en otra columna. 
				\item Se utiliza el comando hline para pasar a otra fila.
			\end{enumerate}
			\item Alineamiento: Con begin$\{$align$\}$ y end $\{$align$\}$.
		\end{enumerate}
		\item Cosas vistas en Python:
		\begin{enumerate}
			\item Resolvimos el Problema 1 de la Tarea 4 en clase, con una sucesión recursiva.
			\item getwd : cumple la misma función que el comando pwd en la terminal. 
			\item listir : cumple la misma función que ls en la terminal. 
			\item chdir : cumple la misma función que el comando cd en la terminal. 
		\end{enumerate}
	\end{enumerate}
\end{document}
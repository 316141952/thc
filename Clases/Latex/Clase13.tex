\documentclass{article} %El formato de documento que utilizo es de artículo.
\usepackage{amsmath} %Este paquete mejora la información estructural y la visualización de documentos que contengan fórmulas matemáticas.
\usepackage{amssymb} %Este paquete provee una colección extendida de símbolos matemáticos.
\usepackage{graphicx} %Este paquete es para poder insertar imágenes
\usepackage[utf8]{inputenc} %Este paquete es para los carácteres especiales en español, como los acentos.
\graphicspath{{/home/Valeria/thc/Clases/Latex/Imagenes}}
\renewcommand*\ttdefault{cmvtt}
\renewcommand*\familydefault{ttdefault}
\usepackage[T1]{fontenc}

\title{\Huge Taller de Herramientas Computacionales}
\author{\huge Valeria Ortiz Cervantes}
\date{\LARGE 23 de enero del 2019}

\begin{document}
\maketitle
\begin{center}
	\subsection*{\LARGE Universidad Nacional Autónoma de México.\\Facultad de Ciencias.\\}
	\includegraphics[scale=3]{/1.jpg}
\end{center}
\newpage
\title{\LARGE Clase XIII. Bitácora\\}
\text{En la clase de hoy solo vimos cosas en Python, y estas fueron las cosas nuevas que vimos:}
\begin{enumerate}
	\item if a: : esto es equivalente a poner if bool(a):
	\item , : una coma después de un print(), hace que se omitan los saltos de línea. 
	\item Funciones recursivas: las funciones recursivas son funciones que se llaman a si mismas.\\ Estos son los ejemplos que realizamos en clase :
	\begin{verbatim}
	Ejemplo 1: esta es una función que calcula el enésimo término de la suceción 
	de Fibonacci de manera recursiva.
	def fib (n) :
	    ''' Calcula el enésimo término de la
	sucesión de Fibonacci con n natural '''
	    if n > 2:
	        return fib(n-1) + fib (n-2)
	    else :
	        return 1
	
	Ejemplo 2: esta es una función que calcula la suma de los primeros n números 
	naturales de manera recursiva.
	def suma (x) :
	    if x == 1 :
	        return 1
	    else :
	        return x + suma (x-1)
	
	Ejemplo 3: esta es una función recursiva que al introducir una lista como entrada 
	te devuelve los valores de la lista pero sin estar en ella.
	def recursiva (L) :
	    n = len (L)
	    if n > 1 :
	        print L[0],
	        recursiva (L[1:])
	    else :
	        print L[0]
	
	Ejemplo 4: esta función recursiva hace lo mismo que la anterior pero la modificamos 
	para que sea más corta.
	def printr (L) :
	    if L:
	        print L [0],
	        printr (L [1:])
	    else :
	        None
	\end{verbatim}
	\item Tipos de variables:
	\begin{enumerate}
		\item Local: variable que se encuentra dentro de una función, al estar dentro de una función sólo es válida dentro de la misma. 
		\item Global: variable que se encuentra fuera de una función, esta es válida tanto dentro como fuera de funciones. 
	\end{enumerate}
\end{enumerate}
\end{document}